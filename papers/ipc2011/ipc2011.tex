\documentclass{article}
\usepackage{url}
\usepackage{aaai}


\def\BibTeX{Bib\TeX}
\parindent=0pt
\parskip=\baselineskip

\begin{document}

\title{Optimize-and-Learn: an Algorithm Framework used for Parameter Tuning of the Divide-and-Evolve Method applied to AI PLanning}


\author{M{\'a}ty{\'a}s Brendel \\ \\ Projet TAO \\ INRIA Saclay \& LRI\\ Universit{\'e} Paris Sud \\ Orsay, France\\ matthias.brendel@lri.fr   
\And Marc Schoenauer \\ \\ Projet TAO \\ INRIA Saclay \& LRI\\ Universit{\'e} Paris Sud \\ Orsay, France\\ marc.schoenauer@inria.fr}



\date{February 11, 2011}
\maketitle
\begin{abstract}
\noindent Learn-and-Optimize ~(LaO) ~is ~a ~generic ~surrogate ~based method for parameter tuning combining learning and optimization. In this paper LaO is used to tune Divide-and-Evolve (DaE), an Evolutionary Algorithm for AI Planning. The LaO framework makes it possible to learn the relation between some features describing a given instance and the optimal parameters for this instance, thus it enables to extrapolate this relation to unknown instances in the same domain. Moreover, the learned model is used as a surrogate-model to accelerate the search for the optimal parameters. It hence becomes possible to solve intra-domain and extra-domain generalization in a single framework. The proposed implementation of LaO uses an Artificial Neural Network for learning the mapping between features and optimal parameters, and the Covariance Matrix Adaptation Evolution Strategy for optimization. 
\end{abstract}

\section{Introduction}

Parameter tuning is basically a general optimization problem applied off-line to find the best parameters for complex algorithms, for example for Evolutionary Algorithms (EAs). Whereas the efficiency of EAs has been demonstrated on several application domains \cite{practice08,ParameterSettingBook07}, they usually need computationally expensive parameter tuning. Consequently, one is tempted to use either the default parameters of the framework he is using, or parameter values given in the literature for problems that are similar to his one. 

there are as many parameter tuning algorithms as optimization techniques \cite{Eiben2007}, \cite{Montero:2010}. However, several specialized methods have been proposed, and the most prominent today are Racing \cite{birattari2002}, REVAC \cite{Nannen07}, SPO \cite{SPO:CEC05}, and ParamILS \cite{ParamILS-JAIR}. All these approaches face the same crucial generalization issue: can a parameter set that has been optimized for a given problem be successfully used to another one? The answer of course depends on the similarity of both problems. However, even in an optimization domain as precisely defined as AI Planning, there are very few results describing meaningful similarity measures between problem instances. 

Unfortunately, until now, nobody has yet proposed a set of features for AI Planning problems in general, that would be sufficient to describe the characteristics of a problem, like it was done in the SAT domain \cite{Hutter06}. This paper makes a step toward a framework for parameter tuning applied generally for AI Planning and proposes a preliminary set of features. The Learn-and-Optimize (LaO) framework consists of the combination of optimizing (i.e., parameter tuning) and learning, i.e., finding the mapping between features and best parameters. Furthermore, the reuslts of learning will already be useful during further the optimization phases, using the learned model as in standard surrogate-model based techniques.

In this paper, the target optimization technique is Evolutionary Algorithms (EA), more precisely the evolutionary AI planner called Divide-and-Evolve (DaE). However, DaE will be here considered as a black-box algorithm, without any modification for the purpose of this work than its original version described in \cite{BibEvoCop:2010}. 

The paper is organized as follows: Section \ref{section:dae} describes the evolutionary  Divide-and-Evolve algorithm. Section \ref{section:general} introduces the original, top level parameter tuning method, Learn-and-Optimize. Finally, \ref{section:implementation} describes the implementation of the framework used in the Planning and Learning Part of IPC2011.

\section{Divide-and-Evolve}
\label{section:dae}

Divide-and-Evolve (DaE) -- a novel hybridization of Evolutionary Algorithms (EAs) with AI Planning -- has been first proposed in \cite{DAE:EvoCOP06}. For a complete formal description, see \cite{Bibai:ICAPS2010}.

The basic idea of DaE in order to solve a planning task ${\cal P}_D(I,G)$ is to find a sequence of states $S_1, \ldots, S_n$, and to use some embedded planner to solve the series of planning problems ${\cal P}_D(S_{k},S_{k+1})$, for $k \in [0,n]$ (with the convention that $S_0 = I$ and $S_{n+1} = G$). The generation and optimization of the sequence of states $(S_i)_{i \in [1,n]}$ is driven by an evolutionary algorithm. 
The fitness (quality criterion) of a list of partial states $S_1, \ldots,$ $S_n$ is computed by repeatedly calling the external 'embedded' planner to solve the sequence of problems ${\cal P}_D(S_{k},S_{k+1})$, $\{k=0,\ldots,n\}$. The concatenation of the corresponding plans (possibly with some compression step) is a solution of the initial problem.
Any existing planner can be used as embedded planner, but since guaranty of optimality at all calls is not mandatory in order for DaE to obtain good quality results \cite{Bibai:ICAPS2010}, a sub-optimal, but fast planner is used: YAHSP \cite{V:icaps04} is a lookahead strategy planning system for sub-optimal planning which uses the  actions in the relaxed plan to compute reachable states in order to speed up the search process. 

A state is a list of atoms built over the set of predicates and the set of object instances. However, searching the space of complete states would result in a rapid explosion of the size of the search space. Moreover, goals of planning problem need only be to defined as partial states. It thus seems more practical to search only sequences of partial states, and to limit the choice of possible atoms used within such partial states. However, this raises the issue of the choice of the atoms to be used to represent individuals, among all possible atoms. The result of the previous experiments on different domains of temporal planning tasks from the IPC benchmark series \cite{BibEvoCop2009} demonstrates the need for a very careful choice of the atoms that are used to build the partial states. 

The method used to build the partial states is based on an estimation of the earliest time from which an atom can become true. Such estimation can be obtained by any admissible heuristic function (e.g $h^1,h^2...$ \cite{HaslumGeffner-AIPS-2000}). The possible start times are then used in order to restrict the candidate atoms for each partial state. A partial state is built at a given time by randomly choosing among several atoms that are possibly true at this time. The sequence of states is then built by preserving the estimated chronology between atoms (time consistency).

An individual in DaE is hence represented as a variable-length ordered time-consistent list of partial states, and each state is a variable-length list of atoms that are not pairwise mutex, as far as the initial grounding of all atoms  can tell (exactly determining if two atoms are mutex amounts to solving a complete planning problem). Furthermore, all operators that manipulate the representation (see below) maintain the chronology between atoms and the approximate local consistency of a state, i.e. avoid pairwise mutexes.

One-point crossover is used, adapted to variable-length representation in that both crossover points are independently chosen, uniformly in both parents.
Four different mutation operators have been designed, and once an individual has been chosen for mutation (according to a population-level mutation rate), the choice of which mutation to apply is made according to user-defined relative weights. 
Because an individual is a variable length list of states, and a state is a variable length list of atoms, the mutation 
operator can act at both levels: at the individual level by adding (addState) or removing (delState) 
a state; or at the state level by adding (addAtom) or removing (delAtom) some atoms in the given state. 
The complete list of DaE parameters that required some tuning is given in Table \ref{table:parameters}.

\section{The General LaO Framework}
\label{section:general}

In Parameter Tuning applied to AI PLanning tuning one instance has of course a sense if only that instance is to be solved. Parameters tuned for one instance however, may not be optimal for other instances, as \cite{BibGECCO:2010} demonstrates it. This very same paper also demonstrates that global tuning for several domains is even more inferior.

Our Learn-and-Optimize framework (LaO) aims to answer this question and to solve intra-domain generalization problems, by adding learning to optimization. We may assume that there might be a relation between some features of the instance and the optimal parameters. If we could learn such a relation representable by a mapping, we could account both for intra- and extra-domain variability of the optimal parameters. The features could describe differences both between instances from the same domain, both differences between domains. To do this, we try to extract features both from the domain-file and the instance-file.

Suppose we have n features and m parameters. For the sake of simplicity and generality, both the fitness value, the features and the parameters are considered as real values. Parameter tuning is the optimization (in our case minimization) of the fitness function \begin{math}f:\mathbf{R}^m\to \mathbf{R} \end{math}, which function is different for each instance. In our case f is the expected value off the stochastic algorithm DaE executed with parameter \begin{math} p \in \mathbf{R}^m \end{math}. The optimal parameter is defined by \begin{math} p_{opt}=argmin_p\{f(p)\} \end{math}. For each instance there is a set \begin{math} F \in \mathbf{R}^n \end{math}, the features of that instance and the corresponding domain. 

We suppose that there exists an "almost unambiguous" mapping from the feature space to the optimal parameter space. 

\begin{equation} p_F: \mathbf{R}^n \to \mathbf{R}^m, p_F(F)=p_{opt} \end{equation}.	

The relation \begin{math} p_F \end{math} between features and optimal parameters can be learned by any supervised learning method capable of representing, interpolating and extrapolating  \begin{math}\mathbf{R}^n\to \mathbf{R}^m \end{math} mappings. The learning method shall also be capable to resolve the unambiguity of the mapping.

A simple method could be developed by using any known parameter tuning method for an appropriate training set of instances in a domain, and then use an appropriate supervised learning method to learn the relationship of the features and the best parameters. However, we can do more: learning and optimizing may be combined, and thus we get our LaO algorithm.

It is not a new idea to use some kind of model in optimization. Several so called surrogate-model based optimization methods exist. In our case, however, there is a difference that we have several instances to optimize, we have only one model, and that model is actually a mapping from the feature-space to the parameter-space. Nevertheless, there is no question about how to use such a model of \begin{math}p_F\end{math} in optimization: one can always ask the model for hints of parameters. Naturally, if the model was perfectly fit to the training data, it would be of no use, since it would return the same hint as trained. Therefore under-fitting is beneficial during the optimization phase to get new hints. One shall of course avoid over-fitting also in the end, but in the end we can not allow under-fitting.

It seems most reasonable that the stopping criterion of LaO is determined by the stopping criterion of the optimizer algorithm. After exiting one can also do a re-training of the learner with the best parameters found.

\section{An Implementation of LaO}
\label{section:implementation}

Our choice for supervised learning method was a multilayer Feed-Forward Artificial Neural Network (ANN), but other algorithms may also be used. We can suppose that the relation \begin{math}p_F\end{math} is not very complex, which means that a simple ANN may be used. One mapping shall be trained for one domain. One can also try to train a single domain-independent ANN, but that will be left to the future.

The only open decision is which optimizer to use for parameter tuning. Since we have several instances to run, we can only afford a simple optimizer, therefore simply a Covariance Matrix Adaptation Evolution Strategy was chosen, specifically, a (1+1)-CMA-ES (in short, CMA-ES, see \cite{hansen2001ecj}). We started CMA-ES with a known set of good default parameters, taken from \cite{BibGECCO:2010}.

There is one element added to a conventional CMA-ES, and this is gene-transfer between instances. We have one (1+1)-CMA-ES running for each instance, because we can not afford to use a larger population for a single instance. However, the (1+1)-CMA-ES instances of all the instances form a set of individuals. Crossover between individuals is usually not used in ES, however, in our case a good chromosome of one instance may at least help another instance. Thus it may be used as a hint in the optimization. Therefore random gene-transfer was used in our algorithm. When the Genetransferer is requested for a hint for one instance, it returns with uniform random distribution the so-far best parameter of a different instance. Naturally, the default parameters are not tried twice.

There is one additional technical problem with CMA-ES, this is that each parameter is restricted to an interval. This seems reasonable and makes our algorithm more stable. The parameters are actually normalized linearly to the [0,1] interval. For the boundary problem we applied a simple version of the box constraint handling technique described in \cite{hansen2009tec}. Our penalty term was simply \begin{math}||p^{feas}-p|| \end{math}, where \begin{math}p^{feas}\end{math} is the closest value in the box. Moreover, only \begin{math}p^{feas}\end{math} was recorded as a feasible solution to pass to the ANN. Note that the Genetransferer and the ANN itself can not give hints outside of the box. In order to not to compromise too much CMA-ES several iterations of this were carried out for one hint of the ANN and one gene-transfer.

Our realization of our LaO algorithm uses the Shark library (\cite{shark08}) for CMA-ES and the FANN library for ANN (\cite{nissen}). To evaluate each parameter-setting with each instance, we use a computer-cluster with approximately 60 computers, most of them have 4 cores, but some have 8. The cluster is used by many researchers, therefore our algorithm contains a scheduler to use the free capacity on this cluster.

Since the parallel architecture used in our algorithm is somewhat heterogeneous, we can not rely on a fixed running time, which depends on the hardware. Therefore, for each evaluation the number of evaluations is fixed for DaE. Note that the number of YASHP evaluations is approximately proportional with running time, so that the execution time for a particular computer is also determined independently of the parameter-settings. For example, even if the size of the population is increased, with fixed number of evaluation the number of generations will be limited respectively resulting approximatively in the same running time for each parameter-settings. 

Moreover, since DaE is not deterministic, we carried out 11 runs and took the median fitness-value as the result for that instance and that parameter-settings.


\section{Acknowledgements}
This work is funded through French ANR project DESCARWIN ANR-09-COSI-002.\\~\\
% \section{Restrictions on Further Use}

\bibliographystyle{aaai}
\bibliography{ipc2011}

\end{document}






