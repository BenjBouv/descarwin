% This is "sig-alternate.tex" V1.9 April 2009
% This file should be compiled with V2.4 of "sig-alternate.cls" April 2009
%
% This example file demonstrates the use of the 'sig-alternate.cls'
% V2.4 LaTeX2e document class file. It is for those submitting
% articles to ACM Conference Proceedings WHO DO NOT WISH TO
% STRICTLY ADHERE TO THE SIGS (PUBS-BOARD-ENDORSED) STYLE.
% The 'sig-alternate.cls' file will produce a similar-looking,
% albeit, 'tighter' paper resulting in, invariably, fewer pages.
%
% ----------------------------------------------------------------------------------------------------------------
% This .tex file (and associated .cls V2.4) produces:
%       1) The Permission Statement
%       2) The Conference (location) Info information
%       3) The Copyright Line with ACM data
%       4) NO page numbers
%
% as against the acm_proc_article-sp.cls file which
% DOES NOT produce 1) thru' 3) above.
%
% Using 'sig-alternate.cls' you have control, however, from within
% the source .tex file, over both the CopyrightYear
% (defaulted to 200X) and the ACM Copyright Data
% (defaulted to X-XXXXX-XX-X/XX/XX).
% e.g.
% \CopyrightYear{2007} will cause 2007 to appear in the copyright line.
% \crdata{0-12345-67-8/90/12} will cause 0-12345-67-8/90/12 to appear in the copyright line.
%
% ---------------------------------------------------------------------------------------------------------------
% This .tex source is an example which *does* use
% the .bib file (from which the .bbl file % is produced).
% REMEMBER HOWEVER: After having produced the .bbl file,
% and prior to final submission, you *NEED* to 'insert'
% your .bbl file into your source .tex file so as to provide
% ONE 'self-contained' source file.
%
% ================= IF YOU HAVE QUESTIONS =======================
% Questions regarding the SIGS styles, SIGS policies and
% procedures, Conferences etc. should be sent to
% Adrienne Griscti (griscti@acm.org)
%
% Technical questions _only_ to
% Gerald Murray (murray@hq.acm.org)
% ===============================================================
%
% For tracking purposes - this is V1.9 - April 2009

\documentclass{sig-alternate}

\newcommand{\dae}{{\em Divide-and-Evolve}}
\newcommand{\DAEX}{{\sc DaE$_{\text{X}}$}}
\newcommand{\DAEYAHSP}{{\sc DaE$_{\text{YAHSP}}$}}
\newcommand{\YAHSP}{{\sc YAHSP}}

\def\UU{{\mathbb{U}}}


\begin{document}

%\title{A Multicore Implementation of the Divide-and-Evolve Algorithm}
%\title{Speeding Up the Individual Evaluation of the Divide-and-Evolve Algorithm}
%\title{Multithreading the Divide-and-Evolve Algorithm for a Multicore Machine}
\title{A Multithreaded Release of the Divide-and-Evolve Planning System for a Multicore Machine}

%
% You need the command \numberofauthors to handle the 'placement
% and alignment' of the authors beneath the title.
%
% For aesthetic reasons, we recommend 'three authors at a time'
% i.e. three 'name/affiliation blocks' be placed beneath the title.
%
% NOTE: You are NOT restricted in how many 'rows' of
% "name/affiliations" may appear. We just ask that you restrict
% the number of 'columns' to three.
%
% Because of the available 'opening page real-estate'
% we ask you to refrain from putting more than six authors
% (two rows with three columns) beneath the article title.
% More than six makes the first-page appear very cluttered indeed.
%
% Use the \alignauthor commands to handle the names
% and affiliations for an 'aesthetic maximum' of six authors.
% Add names, affiliations, addresses for
% the seventh etc. author(s) as the argument for the
% \additionalauthors command.
% These 'additional authors' will be output/set for you
% without further effort on your part as the last section in
% the body of your article BEFORE References or any Appendices.

\numberofauthors{4}

\author{
\alignauthor Caner Candan\\
\affaddr{\mbox{Thales Research \& Technology}}\\
\affaddr{Palaiseau, France}\\
\email{\normalsize caner.candan@thalesgroup.com}
\alignauthor Johann Dr{\'e}o\\
\affaddr{\mbox{Thales Research \& Technology}}\\
\affaddr{Palaiseau, France}\\
\email{\normalsize johann.dreo@thalesgroup.com}
\and
\alignauthor Pierre Sav{\'e}ant\\
\affaddr{\mbox{Thales Research \& Technology}}\\
\affaddr{Palaiseau, France}\\
\email{\normalsize pierre.saveant@thalesgroup.com}
\alignauthor Vincent Vidal\\
\affaddr{ONERA -- DCSD}\\
\affaddr{Toulouse, France}\\
\email{\normalsize vincent.vidal@onera.fr}
}

%\numberofauthors{2}
%\author{
%\alignauthor
%Caner Candan\\
%Johann Dr{\'e}o\\
%Pierre Sav{\'e}ant\\
%\affaddr{Thales Research \& Technology}\\
%\affaddr{Palaiseau, France}\\
%\email{first.last@thalesgroup.com}
%\alignauthor
%Vincent Vidal\\
%\affaddr{ONERA -- DCSD}\\
%\affaddr{Toulouse, France}\\
%\email{Vincent.Vidal@onera.fr}
%}

%\numberofauthors{2}
%\author{
%\alignauthor
%Caner Candan$^{1}$ ~~~~~~~~ Johann Dr{\'e}o$^{1}$\\
%\affaddr{$^{1}$Thales Research \& Technology}\\
%\affaddr{Palaiseau, France}\\
%\email{first.last@thalesgroup.com}
%\alignauthor
%Pierre Sav{\'e}ant$^{1}$ ~~~~~~~~ Vincent Vidal$^{2}$\\
%\affaddr{$^{2}$ONERA -- DCSD}\\
%\affaddr{Toulouse, France}\\
%\email{Vincent.Vidal@onera.fr}
%}

%\numberofauthors{4}
%\author{
%\alignauthor
%Caner Candan\\
%\affaddr{Thales Research \& Technology}\\
%\affaddr{Palaiseau, France}\\
%\email{first.last@thalesgroup.com}
%\alignauthor
%Johann Dr{\'e}o\\
%\alignauthor
%Pierre Sav{\'e}ant\\
%\alignauthor
%Vincent Vidal\\
%\affaddr{ONERA -- DCSD}\\
%\affaddr{Toulouse, France}\\
%\email{Vincent.Vidal@onera.fr}
%}

\maketitle
\begin{abstract}
This paper provides 
\end{abstract}

\section{Introduction}

$\UU$ denotes a uniform random draw from the set given as argument.

\cite{paradiseo:JHeuristics2004}
\cite{paradiseo:ParallelComputing2004}
\cite{alba:IEEE2002}
\cite{alba:COR2008}
\cite{alba:IPL2002}
\cite{burns:JAIR2010}
\cite{burns:icaps2009}
\cite{burns:ijcai2009}
\cite{vidal:socs2010}
\cite{dae:icaps2010}
\cite{dae:evocop2006}

\section{Discussion and Conclusion}

%ACKNOWLEDGMENTS are optional
%\section{Acknowledgments}

%
% The following two commands are all you need in the
% initial runs of your .tex file to
% produce the bibliography for the citations in your paper.
\bibliographystyle{abbrv}
\bibliography{dae_mt}  % sigproc.bib is the name of the Bibliography in this case
% You must have a proper ".bib" file
%  and remember to run:
% latex bibtex latex latex
% to resolve all references
%
% ACM needs 'a single self-contained file'!
%
%Generated by bibtex from your ~.bib file.  Run latex,
%then bibtex, then latex twice (to resolve references)
%to create the ~.bbl file.  Insert that ~.bbl file into
%the .tex source file and comment out
%the command \texttt{{\char'134}thebibliography}.
\end{document}
