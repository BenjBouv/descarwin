\documentclass[letterpaper]{article}
\usepackage{aaai}
\usepackage{times}
\usepackage{helvet}
\usepackage{courier}
\usepackage{amsmath}
\usepackage{amsfonts}
\usepackage{amssymb}
\usepackage{graphicx}
% %%%%%%%%%%%%%%%%%%%%%%%%%%%%%%%%%%%%%%%%%%%%%%%%%%%%%%
% PDFMARK for TeX and GhostScript
% Uncomment and complete the following for metadata if
% your paper is typeset using TeX and GhostScript (e.g
% if you use .ps or .eps files in your paper):
% \special{! /pdfmark where
% {pop} {userdict /pdfmark /cleartomark load put} ifelse
% [ /Author (John Doe, Jane Doe)
% /Title (Paper Title)
% /Keywords (AAAI, artificial intelligence)
% /DOCINFO pdfmark}
% %%%%%%%%%%%%%%%%%%%%%%%%%%%%%%%%%%%%%%%%%%%%%%%%%%%%%%
% PDFINFO for PDFTeX
% Uncomment and complete the following for metadata if
% your paper is typeset using PDFTeX
% \pdfinfo{
% /Title (Input Your Title Here)
% /Subject (Input The Proceedings Title Here)
% /Author (First Name, Last Name;
% First Name, Last Name;
% First Name, Last Name;)
% }
% %%%%%%%%%%%%%%%%%%%%%%%%%%%%%%%%%%%%%%%%%%%%%%%%%%%%%%
% Uncomment only if you need to use section numbers
% and change the 0 to a 1 or 2
% \setcounter{secnumdepth}{0}
% %%%%%%%%%%%%%%%%%%%%%%%%%%%%%%%%%%%%%%%%%%%%%%%%%%%%%%

\title{Divide-and-Evolve: the marriage of Descartes and Darwin\thanks{This work is being partially funded by Agence Nationale de la Recherche under research contract ANR-09-COSI-002-01}}
% Please leave SVN version number $Revision: 1131 $

%\author{Blind submission \#14}

\author{Johann Dr{\'e}o \ \ \ \ \ \ Pierre Sav{\'e}ant\\Thales Research \& Technology\\Palaiseau, France\\first.last@thalesgroup.com
\And Marc Schoenauer\\INRIA Saclay \& LRI\\Orsay, France\\marc.schoenauer@inria.fr
\And Vincent Vidal\\ONERA -- DCSD\\ Toulouse, France \\ Vincent.Vidal@onera.fr}


%\newcommand{\spread}{\linespread{0.8}}
\newcommand{\spread}{\linespread{1.0}}
\newcommand{\dae}{{\em Divide-and-Evolve}}
\newcommand{\DAEX}{{\sc DaE$_{\text{X}}$}}
\newcommand{\DAEYAHSP}{{\sc DaE$_{\text{YAHSP}}$}}
\newcommand{\YAHSP}{{\sc YAHSP}}

\begin{document}
\maketitle

\begin{abstract}




\end{abstract}


\section{Introduction}
\DAEX,   the   concrete   implementation   of   the   \dae\   paradigm,   is   a
domain-independent satisficing planning system based on Evolutionary Computation
\cite{dae:evocop2006}.   The  basic  principle  is   to   carry   out   a   {\em
Divide-and-Conquer} strategy driven by an evolutionary algorithm. The algorithm
is detailed in \cite{dae:icaps2010} and compared with state-of-the-art planners.
In order to solve a planning task ${\cal P}_D(I,G)$, the basic idea of \DAEX\ is
to find a sequence of states $S_1, \ldots, S_n$, and  to  rely  on  an  embedded
planner $X$ to solve the series of planning tasks  ${\cal  P}_D(S_{k},S_{k+1})$,
for $k \in [0,n]$ (with $S_0 = I$ and $S_{n+1} = G$).  A \DAEX\ individual is  a
sequence of goals which define a sequence of subproblems to be  solved  (a  {\it
decomposition}).  These subproblems are submitted successively  to  an  embedded
planner $X$ and the global solution is obtained after the compression  of  these
intermediate solutions.  The  overall  optimization  process  is  controlled  by
an  evolutionary  algorithm.

The fitness implements a gradient towards feasibility for unfeasible individuals
and a gradient towards optimality for feasible individuals. Feasible individuals
are always preferred to unfeasible ones.  Population initialization as  well  as
variation  operators  are  driven  by  the   critical   path   $h^1$   heuristic
\cite{h1:aips2000} in order to discard inconsistent state  orderings,  and  atom
mutual exclusivity inference in order to discard inconsistent states.  Beside  a
standard one-point crossover for variable length representations, four mutations
have been defined: addition (resp.  removal) of a goal in a  sequence,  addition
(resp.  removal) of an atom in a goal.   The  selection  is  a  comparison-based
deterministic   tournament   of   size   5.

% Parameter tuning
For the sequential release, Darwinian-related parameters of \DAEX\ have been
fixed after some early experiments \cite{dae:evocop2006} whereas parameters
related to the variation operators have been tuned using the Racing method
\cite{dae:gecco2010}.

All experiments were done with \DAEYAHSP: the instantiation of \DAEX\ with the
YAHSP heuristic forward search  solver \cite{yahsp:icaps2004}. % + YAHSP memoing
                                % shared by all searches.  
We added two novelties to the version described in
\cite{dae:icaps2010}.  One important parameter is the maximum number of expanded
nodes allowed to the \YAHSP\ sub-solver which defines empirically what is
considered as an easy problem for \YAHSP. As a matter of fact, the minimum
number of required nodes varies from few nodes to thousands depending of the
planning task.  In the current release this number is estimated during the
population initialization stage. An incremental loop is performed until the
ratio of feasible individuals is over a given threshold or a maximum boundary has
been reached. By default this number is doubled at each iteration until at least
one feasible individual is produced or 100000 has been reached.

\section{Conclusion}

\bibliographystyle{aaai}
\bibliography{dae_mt}


\end{document}









